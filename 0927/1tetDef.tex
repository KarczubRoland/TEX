\documentclass{article}
\usepackage[magyar]{babel}
\usepackage{t1enc}
\usepackage{amsthm}

 \theoremstyle{plain}
\newtheorem{tet}{Tétel}
\newtheorem{lem}[tet]{Lemma}

\theoremstyle{definition}
\newtheorem{defin}{Definíció}

\theoremstyle{remark}
\newtheorem{fld}{Feladat}[section]
\newtheorem*{rem}{Megjegyzés}


\begin{document}

    \begin{defin}
        A derékszögű háromszög...
    \end{defin}
    
    \begin{tet}[Pitagorasz]
        Pitagorasz tétele...
    \end{tet}
    
    \begin{proof}[Pitagorasz tétel bizonyítása]
        Készítsünk két darab... \qedhere
    \end{proof}
    
    \begin{lem}
        A háromszögnek 3 oldala van.
    \end{lem}
    
    
    \section{Első}
    
        \begin{fld}
            I. Első feladat
        \end{fld}
    
        \begin{fld}
            I. Második feladat
        \end{fld}
        
        \begin{fld}
            I. Plusz feladat
        \end{fld}
        
    
    \section{Második}
    
        \begin{fld}
            II. Első feladat
        \end{fld}
    
        \begin{fld}
            II. Második feladat
        \end{fld}
        
        \begin{fld}
            II. Plusz feladat
        \end{fld}
        
        \begin{rem}
            Nem kötelező
        \end{rem}
        
        
        \begin{defin}
            Egy háromszögben...
        \end{defin}
    
        \begin{tet}[Háromszög egyenlőtlenség]
             A háromszög bármely oldalának...
        \end{tet}

\end{document}