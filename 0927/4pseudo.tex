\documentclass{article}
\usepackage[magyar]{babel}
\usepackage{t1enc}
\usepackage{algpseudocode}
\usepackage{algorithm}

\algblockdefx{DoWhile}{Do}{While}{\textbf{do}}[1]{\textbf{while}~(#1)~}

\algcblockdefx[Switch]{Switch}{Case}{Default}[1]{\textbf{case}~#1~:}{\textbf{default}}


\floatname{algorithm}{Algoritmus}
\renewcommand{\listalgorithmname}{Algoritmusok listája}

\begin{document}

\listofalgorithms

\begin{algorithm}
    \begin{algorithmic}[2]
        \Procedure{QUICKSORT}{@A,a,b}
            \Require A írható tömb
            \Require 1 $<$ a $<$ b $<$ Hossz[A] indexek
            \Ensure a-b indextartományt rendezzük
            \If{a=b}
            \State \Return{A}  \Comment{ egyelemű tömb mindig rendezett}
            \Else
            \State FELOSZT(@A,a,b,A(a),@q) \Comment{k=A(a), a tartomány első eleme}
            \State QUICKSORT(@A,a,q)
            \State QUICKSORT(@A,q+1,b)
            \State \Return{A}
            \EndIf
        \EndProcedure
    \end{algorithmic}
    \caption{Gyorsrendezés}
\end{algorithm}

\begin{algorithmic}
    \Do
    \State x-1
    \While{x>0}
\end{algorithmic}

\begin{algorithmic}
    \Switch{a}
    \Case{a<0}
    \State a negatív
    \Case{a=0}
    \State a nulla
    \Case{a<0}
    \State a pozitív
    \Default
    \State nem eldönthető
\end{algorithmic}


\end{document}