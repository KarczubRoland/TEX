\documentclass{article}
\usepackage[magyar]{babel}
\usepackage{t1enc}
\usepackage{xcolor}
\usepackage{amsmath}
\usepackage{amssymb}
\usepackage{mathtools}

\begin{document}

    Tekintsük az $L = \{0, 1\}$ halmazt, és rajta a következő, igazságtáblával definiált műveleteket:
    \[
    \begin{array}{c||c}
        x & \overline{x} \\ \hline
        0 & 1 \\
        1 & 0
    \end{array}
    \quad
    \begin{array}{cc||c|c|c}
        x & y & x \vee y  & x \wedge y  & x \rightarrow y \\ \hline
        0 & 0 & 0 & 0 & 1 \\
        0 & 1 & 1 & 0 & 1 \\
        1 & 0 & 1 & 0 & 0 \\
        1 & 1 & 1 & 1 & 1 \\
    \end{array}
    \]
    Legyenek $a, b, c, d \in L.$ Belátjuk a következő azonosságot:
        \[ (a \wedge b \wedge c) \rightarrow d = a \rightarrow \bigl(b \rightarrow (c \rightarrow d)\bigr) \]
    A következő azonosságokat bizonyítás nélkül használjuk:
        \[x \rightarrow y = \overline{x} \vee y\]
        \[\overline{x \vee y} = \overline{x} \wedge \overline{y} 
          \quad
          \overline{x \wedge y} = \overline{x} \vee \overline{y}\]
    A (3) bal oldala, (4) felhasználásával
        \[ (a \wedge b \wedge c) \rightarrow d \underset{(4a)}{=} \overline{a \wedge b \wedge c} \vee d \underset{(4b)}{=} (\overline{a} \vee \overline{b} \vee \overline{c}) \vee d. \]
    A (3) jobb oldala, (4a) ismételt felhasználásával
    \begin{align*}
        a \rightarrow (b \rightarrow (c \rightarrow d)) &= \overline{a} \vee \bigl(b \rightarrow (c \rightarrow d)\bigr) \\
        &= \overline{a} \vee \bigl(\overline{b} \vee (c \rightarrow d)\bigr) \\
        &= \overline{a} \vee \bigl(\overline{b} \vee (\overline{c} \vee d)\bigr),
    \end{align*}
    ami a $\vee$ asszociativitása miatt egyenlő (5) egyenlettel.
\end{document} s