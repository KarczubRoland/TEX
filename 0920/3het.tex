\documentclass{article}
\usepackage{graphicx}
\usepackage[inline]{enumitem}
\usepackage{hulipsum}


\begin{document}



\begin{itemize*}[label=-,itemjoin*={    , és}]
\item elso
\item masodik 
\item harmadik

\end{itemize*}

\begin{enumerate}
\item elso-enumerate
\begin{enumerate}
\item masodik-enumerate
\begin{enumerate}
\item harmadik-enumerate
\begin{enumerate}
\item negyedik-enumerate

\end{enumerate}
\end{enumerate}
\end{enumerate}
\item[-] elso-kulon
\item masodik-kulon
\\
\hulipsum[1-3]
\\
\item harmadik-kulon
\end{enumerate}


\begin{description}
\item [cimke] \hulipsum[1-2]
\item \hulipsum[1-2]
\item[HOOOOOOSSSSZZZZUUUU CCCCCIIIIMMMKKKEEE] \hulipsum[1-5]
\end{description}

\includegraphics{kep.jpg}

A kep fejjel lefele:

\includegraphics[angle=180]{kep.jpg}
\caption{Felirat}

A kep felirattal:


\begin { SCfigure }
 \caption { Az univerzum képének ismételt felhasználása.
Ez a felirat a jobb oldalon lesz VAGY NEM} 
\includegraphics  { kep.jpg } 
\end { SCfigure }

\end{document}

